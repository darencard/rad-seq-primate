\title{A new method for genome-wide marker development and typing holds great promise for molecular primatology}
\author{
        Christina M. Bergey
			\and
			Luca Pozzi
			\and
			Todd R. Disotell
			\and
			Andrew S. Burrell
	}
\date{\today}

\documentclass[12pt]{article}

\usepackage{color}
\usepackage{graphics}

\begin{document}
\maketitle

\begin{abstract}
Over the last two decades primatologists have benefited from the use of numerous molecular markers to study various aspects of primate behavior and evolutionary history. However, most of the studies to date have been based on a single locus (usually mitochondrial DNA) or a few nuclear markers (e.g., microsatellites). Unfortunately, the use of such markers is not only unable to successfully address important questions in primate population genetics and phylogenetics (mainly due to the discordance between gene tree and species tree), but their development is often a time-consuming and expensive task. Today, the advent of next-generation sequencing (NGS) allows researchers to generate large amount of genomic data for non model organisms. However, whole genome sequencing is still cost prohibitive for most primate species. In this paper, we introduce a second generation sequencing technique for typing thousands of genome-wide markers for non-model organisms. Restriction site-Associated DNA Sequencing (RAD-Seq) reduces the complexity of the genome and allow cheap and fast discovery of thousands of markers in many individuals. Here, we describe the principles of this technique and we demonstrate its application in five primates, representing some of the major lineages within the order. Finally, we discuss the promise and possible limitations of such method for doing multi-locus phylogenetics and population genetics in primates.
\end{abstract}

\section{Introduction}

In molecular primatology, as in all fields of molecular biology, the development of variable genetic markers is essential for the study of organisms at different levels, from population genetic to phylogeographic to phylogenetic research (Avise, 1994). During the first decades of the discipline, an impediment to researchers was the need to develop and type polymorphic markers in a taxon of interest. The markers that resulted from this time-consuming and expensive task were often uninformative when applied outside of the population used in their design, necessitating further rounds of primer design or microsatellite assays, for example (Davey et al., 2011). Due to the bottleneck caused by the inefficiency of marker discovery, many population genetic or phylogenetic studies in molecular primatology have been based on one or few loci, usually mitochondrial DNA or a few microsatellites (Ting and Sterner, 2012). Such inferences can reliably give the evolutionary history of those particular regions of the genome, but they fail to adequately capture the complete complex history of the population given the mosaic nature of genomic evolution (Degnan and Rosenberg, 2009; Maddison, 1997; Edwards, 2009; Maddison and Knowles, 2006). Adequate resolution depends on high marker density, and until recently that goal has been out of reach for many primate researchers (Edwards, 2009).

The rapidly decreasing costs of DNA sequencing technology have promised revolutionary gains for primatology (Enard and Paabo 2004; Goodman et al 2005; Ting and Sterner, 2012). A primate researcher benefits from the many nearby sequenced and assembled reference genomes in the order, but genomic studies of non-model organisms nevertheless remain difficult. Though the cost of whole genome sequencing has fallen to a level feasible for many researchers' budgets (e.g., Perry et al., 2012), sequencing whole genomes for the tens or hundreds of individuals desired in a typical population genetic study is often prohibitively expensive and quite possibly superfluous (McCormack, et al. 2012). Fortunately, researchers have recently developed techniques that reduce the complexity of the genome and allow discovery and typing of thousands or tens of thousands of genome-wide makers in many individuals in a single step (Davey et al., 2011; McCormack, et al. 2012). RAD-seq is one such simple, inexpensive reduced representation technique which allows for the sequencing of small fragments of the genome adjacent to restriction enzyme cut sites (Baird et al., 2008). These RAD tags, or Restriction-site Association DNA tags, were originally developed for use in microarray hybridization typing (Miller et al., 2007), but an updated protocol substitutes second-generation DNA sequencing to rapidly discover and type SNPs (Baird et al., 2008; Etter et al., 2011). The lack of reliance on a reference genome and applicability to datasets of many individuals make it a promising technique for phylogenetic or population genetic studies in non-model organisms, such as many primates.

In the present study, we summarize the RAD-seq method in brief and note the many and varied applications since its development. We demonstrate the technique in 5 primates: a lemur, New World monkey, Old World monkey, and two apes, and discuss the features that make RAD-seq promising for primatologists.  

\subsection{The RAD-seq Technique}

The following is a summary of the RAD tag library preparation protocol of Etter et al., 2011 (Fig. 1). The RAD-seq library preparation begins when genomic DNA is digested with a restriction enzyme, such as \emph{EcoRI} or \emph{PspXI} (1A). The P1 adapter is then ligated to the fragments, connected to the sticky end at the restriction enzyme cut site. The P1 adapter contains an amplification site for PCR, an Illumina sequencing priming site, and an individual-specific barcode of five basepairs (1B). Once the barcode has been added, fragments from multiple individuals can be pooled (1C), and the DNA is randomly sheared with a sonicator to have a length distribution under 1 kilobase (1D). To select for reads that are suitable for sequencing on the Illumina platform, the sheared samples are size selected via agarose gel electrophoresis, extracting fragments between 300 and 500 bp in length. The second adapter, P2, is a Y adapter meaning its two halves are complementary for only part of their length (1E). It is ligated to the fragments and then the fragments are amplified via PCR (1F). Because the second adapter has divergent ends, the reverse amplification primer is unable to bind until after the forward amplification primer has filled in its complementary sequence. This ensures that only RAD tags ligated to P1 are able to amplify. After few a cycles of PCR to minimize the risk of introducing PCR artifacts or biases, the library is ready for final clean-up, quality control, and sequencing.

\subsection{Previous RAD-seq Studies}

RAD-Seq is an economical and efficient method for SNP discovery and genotyping. Since its first application by Baird and colleagues (2008) on two model organisms – the fungus \emph{Neurospora crassa} and the three-spined stickleback, \emph{Gasterosteus aculeatus} – RAD-seq has been successfully applied to several organisms for which reference genome information was not available. 

The ability of RAD-seq technology to identify thousands of orthologous SNPs across multiple individuals at both intra- and interspecific level makes this technique extremely promising for the study of population structure (Hohenlohe et al., 2010; Emerson et al., 2010; Keller et al., 2012), gene flow and hybridization (Hohenlohe et al., 2011; Keller et al., 2012), phylogeography (Emerson et al., 2010) and phylogeny (Wagner et al., 2012; Rubin et al., 2012). The RAD-tag sequencing approach has been particularly used to generate SNP data to address questions in population genomics. For example, a series of studies conducted by Hohenlohe and colleagues investigated parallel adaptation and hybridization in several species of fish (Hohenlohe et al. 2010, 2011, 2012), while Emerson et al (2010) identified more than 3,700 SNPs for pitcher plant mosquitoes in eastern North America, providing the first phylogeographic study using RAD sequence data. 

Although RAD sequencing is more effective in addressing questions at or below the level of a single species, a few recent studies have used this technique in the analysis of phylogenetic questions. Rubin et al (2012) provided a simulation study in which they investigated the accuracy of RAD-seq data to reconstruct phylogenies in organisms with different population sizes and clade ages (\emph{Drosophila}, mammals, and yeasts). In their study the authors supported the efficiency of RAD-seq data in inferring phylogenies, but they also caution that this approach achieves the best results in younger clades where more orthologous restriction sites are likely to be retained across species. This simulation analysis was confirmed in two recent empirical studies in which a RAD-tag sequencing approach was successfully used to reconstruct phylogenetic relationships in two recent but speciose radiations: the African cyclids (Wagner et al., 2012) and the \emph{Heliconius} butterflies (Nadeau et al., 2012).

\section{Methods}

\paragraph{Library Preparation and Sequencing}

	We digested genomic DNA from 5 primates with \emph{PspXI} (New England Biolabs) and used it to create a multiplexed RAD tag library. Our library preparation method followed that of Etter et al, 2011 with the following modifications: the P1 adapter bottom oligonucleotide was modified to have an overhang corresponding to the cut site of \emph{PspXI}, and a longer P2 adapter suitable for paired end sequencing was used (P2\_top: 5'- /5Phos/GAT CGG AAG AGC GGT TCA GCA GGA ATG CCG AGA CCG ATC AGA ACA A-3'; P2\_bottom: 5'- CAA GCA GAA GAC GGC ATA CGA GAT CGG TCT CGG CAT TCC TGC TGA ACC GCT CTT CCG ATC*T -3'). Individual-specific barcodes contained in the P1 adapter differed by at least three nucleotides. We chose \emph{PspXI} based on the results of \emph{in silico} digestion of the human, rhesus macaque, and baboon reference genomes using custom Perl scripts. We sequenced the prepared library as one 150-cycle paired-end run and one 150-cycle single-end run of an Illumina MiSeq at the NYU Langone Medical Center's Genome Technology Center using a spike-in of 30\% PhiX DNA to control for low diversity in the library at the barcode and restriction sites. Other individuals were sequenced alongside those of the present study. Sequences are available to download from the NCBI Short Read Archive (accession number SRAXXXXXX.X).

\paragraph{Sequence Analysis - Clustering and SNP Discovery}

As input for the clustering analysis, we combined the first read of the paired-end run and the single-end run reads. We demultiplexed, or separated by barcode, sequence reads and excluded from the analysis reads without an expected barcode or an intact restriction enzyme cut site. We also removed reads with any quality scores below 10. Using the program Stacks, we clustered all reads into sets that differed by no more than two basepairs and compared closely related sets to detect orthologous loci and SNPs using a maximum likelihood approach (Catchen et al., 2011). We tallied orthologous SNPs using VCFtools (Danecek et al., 2011).

\paragraph{Sequence Analysis - Assess RAD Tag Coverage}

To assess the RAD tag coverage, we mapped human and chimpanzee reads to the highest quality primate reference genome, that of humans. Again, we excluded reads without an expected barcode or an intact restriction enzyme cut site. We aligned reads to the human reference genome (GRCh37/hg19, Human Genome Consortium 2001) using BWA with default parameters (Li and Durbin, 2009). We separately mapped the single-end and paired-end data and then combined the resultant files after alignment. We removed reads that were unmapped or that had low mapping quality using Picard (http://picard.sourceforge.net) and BamTools (Barnett et al., 2011). 

After performing local realignment around indels with GATK (DePristo et al., 2011), we identified SNPs and short indels using SAMtools mpileup and BCFtools (Li et al., 2009). We required a minimum coverage of 3 reads and a maximum of 100 to call a SNP or an indel at a given location. We tallied orthologous SNPs using VCFtools (Danecek et al., 2011).
	
To assess how many restriction sites were successfully sequenced and to analyze the degree of overlap between multiplexed individuals' datasets, we first found all possible \emph{PspXI} cut sites in the human genome using the oligoMatch utility in the USCS Genome Browser program and created a BED file of all regions 1000 basepairs upstream and downstream. (Meyer et al., 2012). This allowed us to calculate the coverage of these restriction site-associated regions using BEDtools' multiBamCov program (Quinlan and Hall 2010).

\section{Results}

12.3 million sequencing reads with an intact barcode and restriction enzyme cut site could be assigned confidently to one of the five primates in the present study. Roughly 9.1 million of those reads passed quality control filtration and were clustered into stacks. By comparing these stacks and including only SNPs that were present in multiple individuals, our study identified 7,910 SNPs among all samples. Information for each individual is summarized in Table 1. 

\begin{table}[h]
\caption{Clustered Reads Data}
\begin{center}
	\small
	\begin{tabular}{ p{3cm} || l || p{1.75cm} | p{1.75cm} || p{1.75cm} | p{1.75cm} | l }
		\hline
		Taxon                       & \# Reads  & \# Filtered Reads & \# Ind. Stacks & Mean Coverage (SD)  & \# Shared SNPs \\ \hline\hline
		\emph{Microcebus sp.}       & 2,830,832 & 2,025,103       & 248,324         &  7.66 (20.71)  & 13    \\ \hline
		\emph{Cebus sp.}            & 1,946,096 & 1,427,413       & 107,829         & 12.64 (139.08) & 56    \\ \hline
		\emph{Theropithecus gelada} & 1,918,425 & 1,392,709       & 136,657         &  9.70 (17.34)  & 212   \\ \hline
		\emph{Pan troglodytes}      & 2,616,062 & 1,910,560       & 157,775         & 11.50 (42.27)  & 5,886 \\ \hline
		\emph{Homo sapiens}         & 3,032,823 & 2,374,733       & 131,544         & 17.37 (25.30)  & 5,786 \\ \hline
	\end{tabular}
\end{center}
\end{table}

In the human genome, we found 58,172 possible cut sites for \emph{PspXI} and 116,344 possible sequencing sites (two per cut site, one upstream and one downstream). Of those possible location, 111,686 sites (96.00\%) had at least one mapped read present in human and 91,646 sites (78.77\%) in chimpanzee. For 90,022 sites (77.38\%), both chimpanzee and human had at least one read. When we restrict the analysis to sites with at least three reads, 109,098 sites (93.77\%) had sequences in human, 89,628 sites (77.04\%) in chimpanzee, and 86,604 sites (74.44\%) in both. From these data, we found 9,275 SNPs relative to the human reference genome that were present in both chimpanzee and human datasets.

\begin{table}[h]
\caption{Mapped Reads Data}
\begin{center}
	\small
	\begin{tabular}{ p{3cm} || p{1.75cm} | p{1.75cm} || p{1.75cm} | p{1.75cm} | l }
		\hline
		Taxon                  & \# Reads  & \# Filtered Reads & \# Loci $\ge 1$ Read & \# Loci $\ge 3$ Reads & \# SNPs \\ \hline\hline
		\emph{Pan troglodytes} & 3,917,046 & 2,826,643           &  91,646              &  89,628               & 309,703 \\ \hline
		\emph{Homo sapiens}    & 4,542,978 & 3,784,192           & 111,686              & 109,098               &  35,651 \\ \hline
	\end{tabular}
\end{center}
\end{table}


\section{Discussion}

RAD-Seq represents a fast, cheap, and efficient method for SNP discovery even in species for which no genome reference is available. These characteristics make RAD-seq an extremely promising technique for researchers interested in primate population genomics and phylogenetics. There are several advantages in using RAD-seq over other molecular techniques. First, this methodology is quite cheap and requires little labwork. The development of a library can be completed in only two days of labwork and all the different steps can be easily performed in a standard molecular lab. Also, the possibility to multiplex several individuals in the same Illumina run either using standard barcodes or a custom combinatorial indexing method (see Peterson et al., 2012 for details) allows researchers to reduce the number of sequencing runs, decreasing the costs even further (Peterson et al., 2012; Davey et al., 2011; McCormack, et al. 2012). Second, RAD-seq represents a great improvement in the discovery of molecular markers to be used in population genetics and phylogenetics. Previous studies to date have been based on single locus (mainly mitochondrial DNA) or a few tens of loci (microsatellite for population genetics or nuclear loci for phylogenetics). RAD-seq techniques can easily produce thousands of independent SNPs in a single run increasing 100-1000x the amount of data available to researchers. Finally, RAD sequencing can produce a large amount of orthologous SNP data that can be employed in a wide range of studies, including population genomics and demographics (e.g., effective population size estimates, bottlenecks, etc.), gene flow and hybridization between closely related species, species boundaries, phylogeography, and phylogeny especially at the intrageneric level (Hohenlohe et al., 2010, 2011; Emerson et al., 2010; Keller et al., 2012; Wagner et al., 2012; Rubin et al., 2012). In summary, we believe that the use of RAD-seq technology will provide extremely valuable information to study recent radiation within primates and to address some major open questions in primatology.

Despite the great potential of the application of RAD sequencing in primatology, we also need to point out some possible limitations of this technique. Possibly, the main constraint of employing RAD-seq on a large scale within primates is related to the need for high quality DNA in order to build the library. In this study we used DNA extracted from tissue or blood. However, most molecular primatologists are limited in their use of invasive samples, and more often rely on low quality samples such as hair or feces. Although not yet available, in theory, RAD-seq protocols using non invasive samples could be developed. In a recent study, Perry and colleagues (2010) presented a genomic-scale capture protocol to obtain endogenous DNA from primate fecal samples. Capture methods have been also used to obtain low quantity and poor quality DNA from museum specimens (Mason et al., 2011) or even fossils (Burbano et al., 2010; Krause et al., 2010).

Another possible limitation of the RAD-seq approach is the evolutionary time scale of its application. RAD-seq data in fact might be not suitable for comparing very distantly related taxa (Rubin et al., 2012). In their study, Rubin and colleagues showed a negative correlation between phylogenetic accuracy and evolutionary divergence time, suggesting that the age of a clade is a major determinant of the success of the RAD method (Rubin et al., 2012). The deep divergences between taxa in fact decreases the number of discoverable RAD loci for two main reasons: first, restriction sites can change over time, reducing the number of orthologous loci retained across distantly related taxa; second, orthology is more difficult to infer based on sequence similarity when evolutionary divergence is high (Rubin et al., 2012). This correlation between accuracy and divergence time either reduces the number of orthologus loci available for phylogenetic reconstruction or increase the amount of missing data; both scenarios can affect phylogenetic performance reducing the support values in many nodes or supporting different topologies. However, despite this drawback, Rubin and colleagues successfully reconstructed the phylogeny of 12 species of \emph{Drosophila}, with a crown age of 40–60 Mya. This result suggests that RAD-seq data might be informative enough to reconstruct the phylogeny of most lineages within primates (crown age between 65 and 85 Mya – Wilkinson et al., 2011; Perelman et al., 2011; Steiper and Seiffert, 2012).

This study illustrates the value of RAD-seq approach in discovering a large number of independent SNPs that can be used to address many questions in primatology, ranging from population genomics to phylogenetics. Our preliminary study of primates shows the feasibility of this technique across the primate order, even when nearby reference genomes are not available. Future developments in both sequencing technologies and computational tools will address - and most likely overcome - the current limitations of RAD sequencing, making this technique viable for studies of a large number of primate species and populations.

\section{Acknowledgements}

The present study was supported by a Leakey Foundation General Grant. The authors would like to thank the NYU Langone Medical Center's Genome Technology Center for assistance with library preparation and sequencing.

\section{Figure Legends}

\paragraph{Figure 1:} An overview of the RAD-seq library creation protocol and initial analysis steps.

\end{document}
