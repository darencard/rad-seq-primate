\title{RADseq Works in Primates, Dammit.}
\author{
        Christina M. Bergey
			\and
			Andrew S. Burrell
			\and
			Luca S.J. Pozzi
			\and
			Todd, I suppose
	}
\date{\today}

\documentclass[12pt]{article}

\begin{document}
\maketitle

\begin{abstract}
\ldots Blah, blah, blah, RADseq, blah, blah, Cercopithecoidea. \ldots
\end{abstract}

\section{Introduction}
\begin{itemize}
	\item Next-gen sequencing revolution promises gains in primatology
	\item Still expensive
	\item Many genomes, but still tough doing genomics on non-model organisms
	\item What is RADseq?
	\item Previous RADseq studies
	\item Why would it be good for primates
	\item PRESENT STUDY
	\begin{itemize}
		\item We did RADseq on 6 Cercopithecoids
		\item Assessed how well it worked
		\item Show it has promise for primates
	\end{itemize}
\end{itemize}

\section{Methods}

\paragraph{Library Preparation and Sequencing}
\begin{itemize}
	\item 6 animals, table with sources and other info
	\item Etter et al. 2011, though with modifications
	\item How we picked enzyme, PspXI. UCSC oligoMatch
	\item Adapter sequences. Barcodes with at least 3 mismatches. PE adapter
	\item What did it look like on BioA? Size? Concentration?
	\item Sequenced on Illumina MiSeq. 150PE
	\item NYU Langone Medical Center's Genome Technology Center (name right?)
	\item How much actually loaded?
	\item 30\% spike in with PhiX control DNA to control for low diversity library
\end{itemize}

\paragraph{Analysis Pipeline - Mapping to Reference Genomes}
\begin{itemize}
	\item Demultiplex. Must have barcode and restriction site intact.
	\item Analyze reads with FastQC (No need to say this)
	\begin{itemize}
		\item Total sequence bp
		\item Maximum possible sequence depth
		\item Other stats that FastQC gives you
	\end{itemize}
	\item Aligned to rhesus genome using BWA aln
	\begin{itemize}
		\item default parameters
	\end{itemize}
	\item Combine paired-end reads with BWA sampe (No need to say this)
	\item Convert to BAM, sort and index with samtools (No need to say this)
	\item Analyze mapped reads with samtools utilities flagstat and idxstats and bamtools utility 
	\item Post-alignment filtering steps
	\begin{itemize}
		\item Fix mate pair info with Picard
		\item Filter for mapped and paired.
		\item Remove dups with Picard
		\item Add read group info with Picard (No need to say this)
		\item Remove reads with low mapping quality with bamtools
	\end{itemize}
\end{itemize}

\paragraph{Analysis Pipeline - Variant Calling}
\begin{itemize}
	\item Local realignment with GATK
	\item Fix paired end data with Picard (No need to say this)
	\item Call SNPs with samtools. Min coverage = 3, max = 100
	\item Summarize SNP stats with vcf-stats (No need to say this)
	%\item Call consensus sequence with samtools
\end{itemize}

\paragraph{Analysis Pipeline - Analysis of Degree of Overlap}
\begin{itemize}
	\item Calculate coverage of restriction site-associated regions
	\begin{itemize}
		\item Info on targeted intervals
		\begin{itemize}
			\item Total number possible targets in rhesus genome (compare to human too?)
			\item Total possible target BP
		\end{itemize}
		\item How many targets did we hit?
		\begin{itemize}
			\item BEDtools multiBamCoverage for this job
			\item Number and percentage of targets with coverage $\ge 1$
			\item Number and percentage of targets with coverage $\ge N$
		\end{itemize}
	\end{itemize}
	\item Count orthologous SNPs shared between individuals
	\begin{itemize}
		\item VCFtools vcf-compare for this job
	\end{itemize}
\end{itemize}

%\paragraph{Analysis Pipeline - Inferring Phylogeny}
%\begin{itemize}
%	\item Using Stacks? 
%	\item Using method like cichlid people?
%	\item Using method like Rubin et al %http://www.plosone.org/article/info%3Adoi%2F10.1371%2Fjournal.pone.0033394
%\end{itemize}

\section{Results}
\begin{itemize}
	\item Table: 
	\begin{itemize}
		\item Number of reads per animal
		\item Number that passed filtration
		\item Number of loci hit
		\item Number of loci hit with coverage $\ge N$
		\item Number of SNPs
	\end{itemize}
	\item SNP Venn diagram?
	\item Table of overlapping region, orthologous SNP counts
	%\item Phylogenetic tree
\end{itemize}

\section{Discussion}
\begin{itemize}
	\item RADseq is viable tool for researcher interested in primate phylogenetics, pop. gen.
	\item Enzyme choice allows control over coverage, number of individuals, number of loci.
	\item Potential problems with RADseq method
	\item Promise for primatology
\end{itemize}

\section{Acknowledgements}
Acknowledgements


\end{document}
